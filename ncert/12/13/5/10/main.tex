\iffalse
\let\negmedspace\undefined

\let\negthickspace\undefined
\documentclass[journal,12pt,twocolumn]{IEEEtran}
\usepackage{cite}
\usepackage{amsmath,amssymb,amsfonts,amsthm}
\usepackage{algorithmic}
\usepackage{graphicx}
\usepackage{textcomp}
\usepackage{xcolor}
\usepackage{txfonts}
\usepackage{listings}
\usepackage{enumitem}
\usepackage{mathtools}
\usepackage{gensymb}
\usepackage[breaklinks=true]{hyperref}
\usepackage{tkz-euclide} 
\usepackage{listings}
\newtheorem{theorem}{Theorem}[section]
\newtheorem{problem}{Problem}
\newtheorem{proposition}{Proposition}[section]
\newtheorem{lemma}{Lemma}[section]
\newtheorem{corollary}[theorem]{Corollary}
\newtheorem{example}{Example}[section]
\newtheorem{definition}[problem]{Definition}
\newcommand{\BEQA}{\begin{eqnarray}}
\newcommand{\EEQA}{\end{eqnarray}}
\newcommand{\define}{\stackrel{\triangle}{=}}
\theoremstyle{remark}
\newtheorem{rem}{Remark}

%\bibliographystyle{ieeetr}
\begin{document}
%

\bibliographystyle{IEEEtran}


\vspace{3cm}

\title{
%	\logo{
ASSIGNMENT-2
%	}
}
\author{A Varun Naik - EE22BTECH11004}


\maketitle

\newpage

%\tableofcontents

\bigskip

\renewcommand{\thefigure}{\theenumi}
\renewcommand{\thetable}{\theenumi}


\providecommand{\pr}[1]{\ensuremath{\Pr\left(#1\right)}}
\providecommand{\prt}[2]{\ensuremath{p_{#1}^{\left(#2\right)} }}        % own macro for this question
\providecommand{\qfunc}[1]{\ensuremath{Q\left(#1\right)}}
\providecommand{\sbrak}[1]{\ensuremath{{}\left[#1\right]}}
\providecommand{\lsbrak}[1]{\ensuremath{{}\left[#1\right.}}
\providecommand{\rsbrak}[1]{\ensuremath{{}\left.#1\right]}}
\providecommand{\brak}[1]{\ensuremath{\left(#1\right)}}
\providecommand{\lbrak}[1]{\ensuremath{\left(#1\right.}}
\providecommand{\rbrak}[1]{\ensuremath{\left.#1\right)}}
\providecommand{\cbrak}[1]{\ensuremath{\left\{#1\right\}}}
\providecommand{\lcbrak}[1]{\ensuremath{\left\{#1\right.}}
\providecommand{\rcbrak}[1]{\ensuremath{\left.#1\right\}}}
\newcommand{\sgn}{\mathop{\mathrm{sgn}}}
\providecommand{\abs}[1]{\left\vert#1\right\vert}
\providecommand{\res}[1]{\Res\displaylimits_{#1}} 
\providecommand{\norm}[1]{\left\lVert#1\right\rVert}
%\providecommand{\norm}[1]{\lVert#1\rVert}
\providecommand{\mtx}[1]{\mathbf{#1}}
\providecommand{\mean}[1]{E\left[ #1 \right]}
\providecommand{\cond}[2]{#1\middle|#2}
\providecommand{\fourier}{\overset{\mathcal{F}}{ \rightleftharpoons}}
\newenvironment{amatrix}[1]{%
  \left(\begin{array}{@{}*{#1}{c}|c@{}}
}{%
  \end{array}\right)
}

\newcommand{\solution}{\noindent \textbf{Solution: }}
\newcommand{\cosec}{\,\text{cosec}\,}
\providecommand{\dec}[2]{\ensuremath{\overset{#1}{\underset{#2}{\gtrless}}}}
\newcommand{\myvec}[1]{\ensuremath{\begin{pmatrix}#1\end{pmatrix}}}
\newcommand{\mydet}[1]{\ensuremath{\begin{vmatrix}#1\end{vmatrix}}}
\newcommand{\myaugvec}[2]{\ensuremath{\begin{amatrix}{#1}#2\end{amatrix}}}
\providecommand{\rank}{\text{rank}}
\providecommand{\pr}[1]{\ensuremath{\Pr\left(#1\right)}}
\providecommand{\qfunc}[1]{\ensuremath{Q\left(#1\right)}}
	\newcommand*{\permcomb}[4][0mu]{{{}^{#3}\mkern#1#2_{#4}}}
\newcommand*{\perm}[1][-3mu]{\permcomb[#1]{P}}
\newcommand*{\comb}[1][-1mu]{\permcomb[#1]{C}}
\providecommand{\qfunc}[1]{\ensuremath{Q\left(#1\right)}}
\providecommand{\gauss}[2]{\mathcal{N}\ensuremath{\left(#1,#2\right)}}
\providecommand{\diff}[2]{\ensuremath{\frac{d{#1}}{d{#2}}}}
\providecommand{\myceil}[1]{\left \lceil #1 \right \rceil }
\newcommand\figref{Fig.~\ref}
\newcommand\tabref{Table~\ref}
\newcommand{\sinc}{\,\text{sinc}\,}
\newcommand{\rect}{\,\text{rect}\,}
\let\vec\mathbf
Question 12.13.5.10 :  A person buys a lottery ticket in 50 lotteries in each of which his chance of winning a prize is $\frac{1}{100}$. What is the probability that he will win a prize
\begin{enumerate}[label=(\alph*)]
 \item  atleast once 
 \item exactly once 
 \item atleast twice ?
\end{enumerate}
\fi
\solution Let X be number of winning prizes in 50 lotteries. The trials are Bernoulli trials.X has binomial distribution with n = 50 and p = $\frac{1}{100}$ 
\begin{table}[h]
\centering
\caption{parameters for CDF}
\label{table.2:ncert/12/13/5/10/}
\begin{tabular}{|c|c|}
\hline
parameter & value\\
\hline
$n$ & 50\\
\hline
$p$ & $\frac{1}{100}$\\
\hline
$q$ & $\frac{99}{100}$\\
\hline
\end{tabular}
\end{table}
\begin{align}
q &= 1 - p = 1-\frac{1}{100} \\
q &=   \frac{99}{100} \\
p_X(k) &= \pr{X=k} \\
p_X(k) &= \comb{n}{k} q^{n-k}p^{k} \\
        &= \comb{50}{k}\brak{\frac{99}{100}}^{50-k}\brak{\frac{1}{100}}^{k} 
\end{align} 

The Cdf for the following pmf :  
\begin{align}
F_X(k) &= \sum_{i=0}^{k} \comb{5}{i}\brak{\frac{99}{100}}^{50-i}\brak{\frac{1}{100}}^{i}
\end{align}
\begin{enumerate}[label=(\alph*)]
\item
  \begin{align}
  \pr{X \geq 1} &= 1 - \pr{X < 1} \\
  &= 1 - F_X(0)  \\
  &= 1 - \comb{50}{0}\brak{\frac{99}{100}}^{50} \\
  &= 0.394
 \end{align}
\item
 \begin{align}
  \pr{X=1} &= \comb{50}{1}\brak{\frac{99}{100}}^{49}\brak{\frac{1}{100}}^{1} \\
  &= 0.3055
\end{align} 
\item
\begin{align}
  \pr{X \geq 2}  &= 1 - \pr{X<2}\\  
   &= 1 - F_X(1) \\
   &= \brak{1 - \frac{99}{100}}^{50} - \frac{1}{2}\brak{\frac{99}{100}}^{49} \\
   &= 0.0894
\end{align}
\end{enumerate}
%\end{document}
