\iffalse
\documentclass{article}
\usepackage{amsmath,amssymb,amsfonts,amsthm}

\begin{document}

\providecommand{\pr}[1]{\ensuremath{\Pr\left(#1\right)}}
\providecommand{\brak}[1]{\ensuremath{\left(#1\right)}}
\newcommand{\solution}{\noindent \textbf{solution: }}

\textbf{Question 11.16.4.9}\\
If 4-digit numbers greater than 5,000 are randomly formed from the digits 0, 1, 3, 5, and 7, what is the probability of forming a number divisible by 5 when:
\begin{enumerate}
    \item The digits are repeated?
    \item The repetition of digits is not allowed?
\end{enumerate}
\fi
\solution 
Let $X$ be a random variable such that:
\begin{align}
	X = \begin{cases}
		0 & n \not\equiv 0 \pmod{5}\\
		1 & n \equiv 0 \pmod{5}\end{cases}
\end{align}

Let $N$ be a 4 digit number $X_{1}X_{2}X_{3}X_{4}$ where $X_{1},X_{2},X_{3},X_{4}$ are digits of the number N.

\begin{table}[h]
    \centering
    %%%%%%%%%%%%%%%%%%%%%%%%%%%%%%%%%%%%%%%%%%%%%%%%%%%%%%%%%%%%%%%%%%%%%%
%%                                                                  %%
%%  This is a LaTeX2e table fragment exported from Gnumeric.        %%
%%                                                                  %%
%%%%%%%%%%%%%%%%%%%%%%%%%%%%%%%%%%%%%%%%%%%%%%%%%%%%%%%%%%%%%%%%%%%%%%

\begin{center}
    \begin{tabular}{|c|c|}
    \hline
	    \textbf{Digit}& \textbf{Position} \\ \hline
	    $X_{1}$ 		   & 	$Thousands's  \ Digit$	\\ \hline
	    $X_{2}$ 		   & 	$Hundred's  \ Digit$	\\ \hline
	    $X_{3}$ 		   & 	$Ten's \ Digit$	\\ \hline
	    $X_{4}$ 		   & 	$One's \ Digit$	\\ \hline
    \end{tabular}
    \end{center}


    \caption{Listing variables}
    \label{table_1}
    \end{table}
\\

Let's solve each part separately. \\

\textbf{(i) Repetition of digits}\\
Let number of favourable outcomes be N(A) and total outcomes be N(T).

For N $>$ 5000,

\begin{table}[h]
    \centering
    %%%%%%%%%%%%%%%%%%%%%%%%%%%%%%%%%%%%%%%%%%%%%%%%%%%%%%%%%%%%%%%%%%%%%%
%%                                                                  %%
%%  This is a LaTeX2e table fragment exported from Gnumeric.        %%
%%                                                                  %%
%%%%%%%%%%%%%%%%%%%%%%%%%%%%%%%%%%%%%%%%%%%%%%%%%%%%%%%%%%%%%%%%%%%%%%

\begin{center}
    \begin{tabular}{|c|c|}
    \hline
	    \textbf{Digit}& \textbf{Favourable} \\ \hline
	    $X_{1}$ 		   & 	$5,7$	\\ \hline
	    $X_{2},X_{3},X_{4}$ 		   & 	${0,1,3,5,7}$ \\ \hline
    \end{tabular}
    \end{center}

    \caption{Conditions for N greater than 5000}
    \label{table_2}
    \end{table}
\\

We must also exclude the case of 5000.
Hence,
\begin{align}
	N(T)=(2\times5\times5\times5)-1 \\
	\implies N(T)=249
\end{align}

\begin{table}[h]
    \centering
    %%%%%%%%%%%%%%%%%%%%%%%%%%%%%%%%%%%%%%%%%%%%%%%%%%%%%%%%%%%%%%%%%%%%%%
%%                                                                  %%
%%  This is a LaTeX2e table fragment exported from Gnumeric.        %%
%%                                                                  %%
%%%%%%%%%%%%%%%%%%%%%%%%%%%%%%%%%%%%%%%%%%%%%%%%%%%%%%%%%%%%%%%%%%%%%%

\begin{center}
    \begin{tabular}{|c|c|}
    \hline
	    \textbf{Digit}& \textbf{Favourable} \\ \hline
	    $X_{1}$ 		   & 	$5,7$	\\ \hline
	    $X_{2},X_{3}$ 		   & 	$0,1,3,5,7$ \\ \hline
	    $X_{4}$ 		   & 	$0,5$ \\ \hline

    \end{tabular}
    \end{center}

    \caption{Conditions for N greater than 5000 and divisible by 5}
    \label{table_3}
    \end{table}
\\

Here also we must exclude the case of 5000.
\begin{align}
	N(A)=(2\times5\times5\times2)-1 \\
	\implies N(A)=99
\end{align}

With this information we can find the required answer,
\begin{align}
	\pr{X=1}=\frac{N(A)}{N(T)}\\
	\implies \pr{X=1}=\frac{33}{83}
\end{align}



\textbf{(ii) No Repetition of Digits}\\
Let number of favourable outcomes be N(B) and total outcomes be N(T).

For N $>$ 5000,

H\begin{table}[h]
    \centering
    %%%%%%%%%%%%%%%%%%%%%%%%%%%%%%%%%%%%%%%%%%%%%%%%%%%%%%%%%%%%%%%%%%%%%%
%%                                                                  %%
%%  This is a LaTeX2e table fragment exported from Gnumeric.        %%
%%                                                                  %%
%%%%%%%%%%%%%%%%%%%%%%%%%%%%%%%%%%%%%%%%%%%%%%%%%%%%%%%%%%%%%%%%%%%%%%

\begin{center}
    \begin{tabular}{|c|c|}
    \hline
	    \textbf{Digit}& \textbf{Favourable} \\ \hline
	    $X_{1}$ 		   & 	$5,7$	\\ \hline
	    $X_{2},X_{3},X_{4}$ 		   & 	$0,1,3,5,7$ \\ \hline

    \end{tabular}
    \end{center}

    \caption{Conditions for N greater than 5000}
    \label{table_4}
    \end{table}
\\
Hence,
\begin{align}
	N(T)=(2\times4\times3\times2) \\
	\implies N(T)=48
\end{align}

For N $>$ 5000 and also divisble by 5:

\begin{align}
	X_{4} = \begin{cases}
		0 & X_{1}=5 \\
		5,0 & X_{1}=7
	\end{cases}
\end{align}

Hence,
\begin{align}
	N(B)=(1\times3\times2\times1)+(1\times3\times2\times2) \\
	\implies N(B)=18
\end{align}

With this information we can find the required answer,
\begin{align}
	\pr{X=1}=\frac{N(B)}{N(T)}\\
	\implies \pr{X=1}=\frac{3}{8}
\end{align}

