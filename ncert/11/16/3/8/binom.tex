\iffalse
\documentclass[journal,12pt,twocolumn]{IEEEtran}
\usepackage{setspace}
\usepackage{gensymb}
\singlespacing
\usepackage[cmex10]{amsmath}
\usepackage{amsthm}
\usepackage{mathrsfs}
\usepackage{txfonts}
\usepackage{stfloats}
\usepackage{bm}
\usepackage{cite}
\usepackage{cases}
\usepackage{subfig}
\usepackage{longtable}
\usepackage{multirow}
\usepackage{enumitem}
\usepackage{mathtools}
\usepackage{tikz}
\usepackage{circuitikz}
\usepackage{verbatim}
\usepackage[breaklinks=true]{hyperref}
\usepackage{tkz-euclide} % loads  TikZ and tkz-base
\usepackage{listings}
\usepackage{color}    
\usepackage{array}    
\usepackage{longtable}
\usepackage{calc}     
\usepackage{multirow} 
\usepackage{hhline}   
\usepackage{ifthen}   
\usepackage{lscape}     
\usepackage{chngcntr}
\DeclareMathOperator*{\Res}{Res}
\renewcommand\thesection{\arabic{section}}
\renewcommand\thesubsection{\thesection.\arabic{subsection}}
\renewcommand\thesubsubsection{\thesubsection.\arabic{subsubsection}}

\renewcommand\thesectiondis{\arabic{section}}
\renewcommand\thesubsectiondis{\thesectiondis.\arabic{subsection}}
\renewcommand\thesubsubsectiondis{\thesubsectiondis.\arabic{subsubsection}}
\renewcommand\thetable{\arabic{table}}
% correct bad hyphenation here
\hyphenation{op-tical net-works semi-conduc-tor}
\def\inputGnumericTable{}                                 %%

\lstset{
%language=C,
frame=single, 
breaklines=true,
columns=fullflexible
}
%\lstset{
%language=tex,
%frame=single, 
%breaklines=true
%}

\begin{document}
\newtheorem{theorem}{Theorem}[section]
\newtheorem{problem}{Problem}
\newtheorem{proposition}{Proposition}[section]
\newtheorem{lemma}{Lemma}[section]
\newtheorem{corollary}[theorem]{Corollary}
\newtheorem{example}{Example}[section]
\newtheorem{definition}[problem]{Definition}
\newcommand{\BEQA}{\begin{eqnarray}}
\newcommand{\EEQA}{\end{eqnarray}}
\newcommand{\define}{\stackrel{\triangle}{=}}
\bibliographystyle{IEEEtran}
\providecommand{\mbf}{\mathbf}
\providecommand{\pr}[1]{\ensuremath{\Pr\left(#1\right)}}
\providecommand{\qfunc}[1]{\ensuremath{Q\left(#1\right)}}
\providecommand{\sbrak}[1]{\ensuremath{{}\left[#1\right]}}
\providecommand{\lsbrak}[1]{\ensuremath{{}\left[#1\right.}}
\providecommand{\rsbrak}[1]{\ensuremath{{}\left.#1\right]}}
\providecommand{\brak}[1]{\ensuremath{\left(#1\right)}}
\providecommand{\lbrak}[1]{\ensuremath{\left(#1\right.}}
\providecommand{\rbrak}[1]{\ensuremath{\left.#1\right)}}
\providecommand{\cbrak}[1]{\ensuremath{\left\{#1\right\}}}
\providecommand{\lcbrak}[1]{\ensuremath{\left\{#1\right.}}
\providecommand{\rcbrak}[1]{\ensuremath{\left.#1\right\}}}
\theoremstyle{remark}
\newtheorem{rem}{Remark}
\newcommand{\sgn}{\mathop{\mathrm{sgn}}}
\providecommand{\abs}[1]{\left\vert#1\right\vert}
\providecommand{\res}[1]{\Res\displaylimits_{#1}} 
\providecommand{\norm}[1]{\left\lVert#1\right\rVert}
\providecommand{\mtx}[1]{\mathbf{#1}}
\providecommand{\mean}[1]{E\left[ #1 \right]}
\providecommand{\fourier}{\overset{\mathcal{F}}{ \rightleftharpoons}}
\providecommand{\system}[1]{\overset{\mathcal{#1}}{ \longleftrightarrow}}
\newcommand{\solution}{\noindent \textbf{Solution: }}
\newcommand{\cosec}{\,\text{cosec}\,}
\providecommand{\dec}[2]{\ensuremath{\overset{#1}{\underset{#2}{\gtrless}}}}
\newcommand{\myvec}[1]{\ensuremath{\begin{pmatrix}#1\end{pmatrix}}}
\newcommand{\mydet}[1]{\ensuremath{\begin{vmatrix}#1\end{vmatrix}}}
\let\vec\mathbf
\def\putbox#1#2#3{\makebox[0in][l]{\makebox[#1][l]{}\raisebox{\baselineskip}[0in][0in]{\raisebox{#2}[0in][0in]{#3}}}}
     \def\rightbox#1{\makebox[0in][r]{#1}}
     \def\centbox#1{\makebox[0in]{#1}}
     \def\topbox#1{\raisebox{-\baselineskip}[0in][0in]{#1}}
     \def\midbox#1{\raisebox{-0.5\baselineskip}[0in][0in]{#1}}

\vspace{3cm}
\title{Probability Assignment}
\author{Gautam Singh}
\maketitle
\bigskip

\begin{abstract}
    This document contains the solution to Question 8 of 
    Exercise 3 in Chapter 16 of the class 11 NCERT textbook.
\end{abstract}

\begin{enumerate}
      \solution 
\fi
		Let the random variable $X$ denote one single coin toss, where 
    obtaining a head is considered a success. Then,
    \begin{align}
        X \sim \textrm{Ber}\brak{p}
        \label{eq:ncert/11/16/3/8/pmf-X}
    \end{align}
    Suppose $X_i, 1\le i\le n$ represent each of the $n$ tosses. Define $Y$ as
    \begin{align}
        Y = \sum_{i=1}^nX_i
        \label{eq:ncert/11/16/3/8/def-Y}
    \end{align}
    Then, since the $X_i$ are iid, the pmf of $Y$ is given by
    \begin{align}
        Y \sim \textrm{Bin}\brak{n,p}
        \label{eq:ncert/11/16/3/8/pmf-Y}
    \end{align}
    The cdf of $Y$ is given by
    \begin{align}
        F_Y\brak{k} &= \pr{Y \le k} \\
                    &=
        \begin{cases}
            0 & k < 0 \\
            \sum_{i=1}^{k}\myvec{n\\i}p^i\brak{1-p}^{n-i} & 1 \le k \le n \\
            1 & k \ge n
        \end{cases}
        \label{eq:ncert/11/16/3/8/cdf-Y}
    \end{align}
    In this case,
    \begin{align}
        p = \frac{1}{2},\ n = 3
    \end{align}
    \begin{enumerate}
        \item We require $\pr{Y=3}$. Thus, from \eqref{eq:ncert/11/16/3/8/pmf-Y},
            \begin{align}
                \pr{Y=3} &= \myvec{n\\3}p^3\brak{1-p}^{n-3} \\
                         &= \frac{1}{8}
                         \label{eq:ncert/11/16/3/8/ans-i}
            \end{align}

        \item We require $\pr{Y=2}$. Thus, from \eqref{eq:ncert/11/16/3/8/pmf-Y},
            \begin{align}
                \pr{Y=2} &= \myvec{n\\2}p^2\brak{1-p}^{n-2} \\
                         &= \frac{3}{8}
            \end{align}

        \item We require $\pr{Y\ge2}$. Since $n = 3$ in \eqref{eq:ncert/11/16/3/8/cdf-Y},
            \begin{align}
                \pr{Y\ge2} &= 1 - \pr{Y<2} \\
                           &= F_Y\brak{3} - F_Y\brak{1} \\
                           &= \sum_{k=2}^3\myvec{n\\k}p^k\brak{1-p}^{n-k} \\
                           &= \frac{1}{2}
            \end{align}

        \item We require $\pr{Y\le2}$. Thus, from \eqref{eq:ncert/11/16/3/8/cdf-Y},
            \begin{align}
                \pr{Y\le2} &= \sum_{k=0}^2\myvec{n\\k}p^k\brak{1-p}^{n-k} \\
                           &= \frac{7}{8}
            \end{align}

        \item We require $\pr{Y=0}$. Thus, from \eqref{eq:ncert/11/16/3/8/pmf-Y},
            \begin{align}
                \pr{Y=0} &= \myvec{n\\0}p^0\brak{1-p}^{n} \\
                           &= \frac{1}{8}
                           \label{eq:ncert/11/16/3/8/ans-v}
            \end{align}

        \item Obtaining 3 tails is the same as obtaining no heads. Hence, from 
        \eqref{eq:ncert/11/16/3/8/ans-v}, we require $\pr{Y=0} = \frac{1}{8}$.

        \item We require $\pr{Y=1}$ (since only one head is obtained). Thus, from 
        \eqref{eq:ncert/11/16/3/8/pmf-Y},
            \begin{align}
                \pr{Y=1} &= \myvec{n\\1}p^1\brak{1-p}^{n-1} \\
                           &= \frac{3}{8}
            \end{align}

        \item We require $\pr{Y=3} = \frac{1}{8}$ from \eqref{eq:ncert/11/16/3/8/ans-i}.

        \item We require $\pr{Y\ge1}$ (since at least one head is obtained). Thus, from 
        \eqref{eq:ncert/11/16/3/8/cdf-Y} and \eqref{eq:ncert/11/16/3/8/ans-v},
            \begin{align}
                \pr{Y\ge1} &= 1 - \pr{Y<1} \\
                           &= 1 - F_Y\brak{0} \\
                           &= 1 - \pr{Y=0} \\
                           &= \frac{7}{8}
            \end{align}
    \end{enumerate}
