\iffalse
\let\negmedspace\undefined
\let\negthickspace\undefined
\documentclass[journal,12pt,twocolumn]{IEEEtran}
\usepackage{cite}
\usepackage{amsmath,amssymb,amsfonts,amsthm}
\usepackage{algorithmic}
\usepackage{graphicx}
\usepackage{textcomp}
\usepackage{xcolor}
\usepackage{txfonts}
\usepackage{listings}
\usepackage{enumitem}
\usepackage{mathtools}
\usepackage{gensymb}
\usepackage{comment}
\usepackage[breaklinks=true]{hyperref}
\usepackage{tkz-euclide} 
\usepackage{listings}
\usepackage{gvv}                                        
\def\inputGnumericTable{}                                 
\usepackage[latin1]{inputenc}                                
\usepackage{color}                                            
\usepackage{array}                                            
\usepackage{longtable}                                       
\usepackage{calc}                                             
\usepackage{multirow}                                         
\usepackage{hhline}                                           
\usepackage{ifthen}                                           
\usepackage{lscape}

\newtheorem{theorem}{Theorem}[section]
\newtheorem{problem}{Problem}
\newtheorem{proposition}{Proposition}[section]
\newtheorem{lemma}{Lemma}[section]
\newtheorem{corollary}[theorem]{Corollary}
\newtheorem{example}{Example}[section]
\newtheorem{definition}[problem]{Definition}
\newcommand{\BEQA}{\begin{eqnarray}}
\newcommand{\EEQA}{\end{eqnarray}}
\newcommand{\define}{\stackrel{\triangle}{=}}
\theoremstyle{remark}
\newtheorem{rem}{Remark}
\begin{document}

\bibliographystyle{IEEEtran}
\vspace{3cm}

\title{GATE: ST - 14.2023}
\author{EE22BTECH11039 - Pandrangi Aditya Sriram$^{*}$% <-this % stops a space
}
\maketitle
\newpage
\bigskip

\renewcommand{\thefigure}{\theenumi}
\renewcommand{\thetable}{\theenumi}


\vspace{3cm}
\textbf{Question:} Consider the probability space $\brak{\Omega, \mathcal{G}, P}$ where $\Omega = [0,2]$ and $\mathcal{G} = \cbrak{\phi, \Omega, [0,1], (1,2]}$. Let $X$ and $Y$ be two functions on $\Omega$ defined as
\begin{align*}
    X(\omega) = 
    \begin{cases}
        1 & \text{if }\omega \in [0, 1]\\
        2 & \text{if }\omega \in (1, 2]
    \end{cases}
\end{align*}
and
\begin{align*}
    Y(\omega) = 
    \begin{cases}
        2 & \text{if }\omega \in [0, 1.5]\\
        3 & \text{if }\omega \in (1.5, 2].
    \end{cases}
\end{align*}
Then which one of the following statements is true?
\begin{enumerate}
    \item [(A)] $X$ is a random variable with respect to $\mathcal{G}$, but $Y$ is not a random variable with respect to $\mathcal{G}$.
    \item [(B)] $Y$ is a random variable with respect to $\mathcal{G}$, but $X$ is not a random variable with respect to $\mathcal{G}$.
    \item [(C)] Neither $X$ nor $Y$ is a random variable with respect to $\mathcal{G}$.
    \item [(C)] Both $X$ and $Y$ are random variables with respect to $\mathcal{G}$.
\end{enumerate} \hfill (GATE ST 2023)\\
\solution
\fi
\begin{enumerate}
    \item For $X$ to be a random variable with respect to $\mathcal{G}$:
    \begin{align}
        X^{-1}(X(\omega)) \in \mathcal{G} && \forall X(\omega) \in \mathcal{T}_1
    \end{align}
    where $\mathcal{T}_1$ is the range of $X(\omega)$.\\
    If $X(\omega) = 1:$
    \begin{align}
        X^{-1}\brak{X(\omega)} &= X^{-1}\brak{1}\\
        &= [0, 1] \\
        &\in \mathcal{G}
    \end{align}
    If $X(\omega) = 2:$
    \begin{align}
        X^{-1}\brak{X(\omega)} &= X^{-1}\brak{2}\\
        &= (1, 2] \\
        &\in \mathcal{G}
    \end{align}
    $\therefore X$ is a random variable with respect to $\mathcal{G}$.

    \item For $Y$ to be a random variable with respect to $\mathcal{G}$:
    \begin{align}
        Y^{-1}(Y(\omega)) \in \mathcal{G} && \forall Y(\omega) \in \mathcal{T}_2
    \end{align}
    where $\mathcal{T}_2$ is the range of $Y(\omega)$.\\
    If $Y(\omega) = 2:$
    \begin{align}
        Y^{-1}\brak{Y(\omega)} &= Y^{-1}\brak{2}\\
        &= [0, 1.5] \\
        &\notin \mathcal{G}
    \end{align}
    $\therefore Y$ is not a random variable with respect to $\mathcal{G}$.
\end{enumerate}