
\let\negmedspace\undefined
\let\negthickspace\undefined
\documentclass[journal,24pt,onecolumn]{IEEEtran}
\usepackage{cite}
\usepackage{amsmath,amssymb,amsfonts,amsthm}
\usepackage{algorithmic}
\usepackage{graphicx}
\usepackage{textcomp}
\usepackage{xcolor}
\usepackage{txfonts}
\usepackage{listings}
\usepackage{enumitem}
\usepackage{mathtools}
\usepackage{gensymb}
\usepackage{comment}
\usepackage[breaklinks=true]{hyperref}
\usepackage{tkz-euclide} 
\usepackage{listings}
\usepackage{gvv}                                        
\def\inputGnumericTable{}                                 
\usepackage[latin1]{inputenc}                                
\usepackage{color}                                            
\usepackage{array}                                            
\usepackage{longtable}                                       
\usepackage{calc}                                             
\usepackage{multirow}                                         
\usepackage{hhline}                                           
\usepackage{ifthen}                                           
\usepackage{lscape}

\newtheorem{theorem}{Theorem}[section]
\newtheorem{problem}{Problem}
\newtheorem{proposition}{Proposition}[section]
\newtheorem{lemma}{Lemma}[section]
\newtheorem{corollary}[theorem]{Corollary}
\newtheorem{example}{Example}[section]
\newtheorem{definition}[problem]{Definition}
\newcommand{\BEQA}{\begin{eqnarray}}
\newcommand{\EEQA}{\end{eqnarray}}
\newcommand{\define}{\stackrel{\triangle}{=}}
\theoremstyle{remark}
\newtheorem{rem}{Remark}
\begin{document}

\bibliographystyle{IEEEtran}
\vspace{3cm}

\title{Exemplar - 12.13.3.50}
\author{EE22BTECH11039 - Pandrangi Aditya Sriram$^{*}$% <-this % stops a space
}
\maketitle
\newpage
\bigskip

\renewcommand{\thefigure}{\theenumi}
\renewcommand{\thetable}{\theenumi}


\vspace{3cm}
\textbf{Question:} The probability distribution of a random variable $X$ is given as under:
\begin{align*}
    p_X(x) =
    \begin{cases}
        kx^2 &\text{for } x = 1,2,3\\
        2kx &\text{for } x = 4,5,6\\
        0 & \text{otherwise}
    \end{cases}
\end{align*}
where k is a constant. Calculate:
\begin{enumerate}
    \item $E\brak{X}$
    \item $E\brak{3X^2}$
    \item $\pr{X \geq 4}$
\end{enumerate}
\solution
From the axiom of total probability,
\begin{align}
    &\sum_{i = 1}^{6} p_X(i) = 1\\
    \implies &\sum_{i = 1}^{3} ki^2 + \sum_{i = 4}^{6} 2ki = 1\\
    \implies &k + 4k + 9k + 8k + 10k + 12k = 1\\
    \implies &k = \frac{1}{44}
\end{align}
Thus, the probability distribution of $X$ is 
\begin{align*}
    p_X(x) =
    \begin{cases}
        \frac{x^2}{44} &\text{for } x = 1,2,3\\
        \frac{2x}{44} &\text{for } x = 4,5,6\\
        0 & \text{otherwise}
    \end{cases}
\end{align*}
\begin{enumerate}
    \item Calculating $E\brak{X}$:
    \begin{align}
        E\brak{X} 
        &= \sum_{i = 1}^{6} ip_X(i)\\
        &= 1\brak{\frac{1}{44}} + 2\brak{\frac{4}{44}} + 3\brak{\frac{9}{44}} + 4\brak{\frac{8}{44}} + 5\brak{\frac{10}{44}} + 6\brak{\frac{12}{44}}\\
        &= \frac{95}{22}\\
        &= 4.32
    \end{align}
    \item Calculating $E\brak{3X^2}$:
    \begin{align}
        E\brak{3X^2} &= 3E\brak{X^2}\\
        &= 3\sum_{i = 1}^{6} i^2 p_{X}(i)\\
        &= 3\brak{1\brak{\frac{1}{44}} + 4\brak{\frac{4}{44}} + 9\brak{\frac{9}{44}} + 16\brak{\frac{8}{44}} + 25\brak{\frac{10}{44}} + 36\brak{\frac{12}{44}}}\\
        &= \frac{2724}{44}\\
        &= 61.91
    \end{align}
    \item Firstly, calculating the CDF:
    \begin{align}
        F_X(x) &= \sum_{i = 1}^{x} p_X(i)\\ &= 
        \begin{cases}
            \sum_{i = 1}^{x} \frac{i^2}{44} & \text{if } x \leq 3\\
            \sum_{i = 1}^{3} \frac{i^2}{44} + \sum_{i = 4}^{x} \frac{2i}{44} & \text{if } x \geq 4
        \end{cases}\\
        &=
        \begin{cases}
            \frac{x(x+1)(2x+1)}{6 \times 44} & \text{if } x \leq 3\\
            \frac{14}{44} + \frac{x(x+1)}{44} - \frac{3 \times 4}{44} & \text{if } x \geq 4
        \end{cases}\\
        &=
        \begin{cases}
            \frac{x(x+1)(2x+1)}{264} & \text{if } x \leq 3\\
            \frac{x(x+1) + 2}{44} & \text{if } x \geq 4
        \end{cases}
    \end{align}
    
    Calculating $\pr{X \geq 4}$:
    \begin{align}
        \pr{X \geq 4} &= 1 - \pr{X \leq 3}\\
        &= 1 - F_X\brak{3}\\
        &= 1 - \frac{3 \times 4 \times 7}{264}\\
        &= \frac{15}{22}\\
        &= 0.68
    \end{align}
\end{enumerate}
\end{document}
